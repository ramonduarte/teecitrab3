%%%%%%%%%%%%%%%%%%%%%%%%%%%%%%%%%%%%%%%%%
% Beamer Presentation
% LaTeX Template
% Version 1.0 (10/11/12)
%
% This template has been downloaded from:
% http://www.LaTeXTemplates.com
%
% License:
% CC BY-NC-SA 3.0 (http://creativecommons.org/licenses/by-nc-sa/3.0/)
%
%%%%%%%%%%%%%%%%%%%%%%%%%%%%%%%%%%%%%%%%%

%----------------------------------------------------------------------------------------
%	PACKAGES AND THEMES
%----------------------------------------------------------------------------------------

\documentclass{beamer}

\mode<presentation> {

% The Beamer class comes with a number of default slide themes
% which change the colors and layouts of slides. Below this is a list
% of all the themes, uncomment each in turn to see what they look like.

%\usetheme{default}
% \usetheme{AnnArbor}
% \usetheme{Antibes} +
% \usetheme{Bergen}
% \usetheme{Berkeley}
% \usetheme{Berlin}
% \usetheme{Boadilla}
\usetheme{CambridgeUS}
% \usetheme{Copenhagen} +
% \usetheme{Darmstadt}
% \usetheme{Dresden}
% \usetheme{Frankfurt}
% \usetheme{Goettingen}
% \usetheme{Hannover}
% \usetheme{Ilmenau}
% \usetheme{JuanLesPins}
% \usetheme{Luebeck} +
% \usetheme{Madrid}
% \usetheme{Malmoe}
% \usetheme{Marburg}
% \usetheme{Montpellier} +
% \usetheme{PaloAlto}
% \usetheme{Pittsburgh}
% \usetheme{Rochester}
% \usetheme{Singapore}
% \usetheme{Szeged}
% \usetheme{Warsaw} +

% As well as themes, the Beamer class has a number of color themes
% for any slide theme. Uncomment each of these in turn to see how it
% changes the colors of your current slide theme.

% \usecolortheme{albatross}
\usecolortheme{beaver}
% \usecolortheme{beetle}
% \usecolortheme{crane}
% \usecolortheme{dolphin}
% \usecolortheme{dove}
% \usecolortheme{fly}
% \usecolortheme{lily}
% \usecolortheme{orchid}
% \usecolortheme{rose}
% \usecolortheme{seagull}
% \usecolortheme{sidebartab}
% \usecolortheme{seahorse}
% \usecolortheme{whale}
% \usecolortheme{wolverine}

%\setbeamertemplate{footline} % To remove the footer line in all slides uncomment this line
%\setbeamertemplate{footline}[page number] % To replace the footer line in all slides with a simple slide count uncomment this line

%\setbeamertemplate{navigation symbols}{} % To remove the navigation symbols from the bottom of all slides uncomment this line
}

\usepackage{graphicx} % Allows including images
\usepackage{booktabs} % Allows the use of \toprule, \midrule and \bottomrule in tables
\usepackage[brazil]{babel}
\selectlanguage{brazil}
\languagepath{brazil}
\deftranslation[to=brazil]{Example}{Exemplo}
\deftranslation[to=brazil]{Theorem}{Teorema}
\deftranslation[to=brazil]{Corollary}{Proposta}
\deftranslation[to=brazil]{Lemma}{Lema}
\deftranslation[to=brazil]{Proof}{Prova}
\usepackage[utf8]{inputenc}
\usepackage{amssymb}
\usepackage{mathtools}
\usepackage{pythonhighlight}

%----------------------------------------------------------------------------------------
%	TITLE PAGE
%----------------------------------------------------------------------------------------

\title[Teorema da Floresta Perfeita]{Duas provas curtas do Teorema da Floresta Perfeita} % The short title appears at the bottom of every slide, the full title is only on the title page

\author{Ramon Duarte de Melo \& Yago Carvalho} % Your name
\institute[UFRJ] % Your institution as it will appear on the bottom of every slide, may be shorthand to save space
{
    Universidade Federal do Rio de Janeiro \\ % Your institution for the title page
    \medskip
    \textit{ramonduarte@poli.ufrj.br \& } \\ % Your email address
    }
    \date{\today} % Date, can be changed to a custom date

\begin{document}

\begin{frame} % SLIDE 1
    \titlepage % Print the title page as the first slide
\end{frame}

\begin{frame} % SLIDE 2
    \frametitle{Sumário} % Table of contents slide, comment this block out to remove it
    \tableofcontents 
\end{frame}

%----------------------------------------------------------------------------------------
%	PRESENTATION SLIDES
%----------------------------------------------------------------------------------------

%------------------------------------------------
% \section{O Problema} 
%------------------------------------------------

\section{Introdução}


\begin{frame} % SLIDE 3
    \frametitle{}

    \begin{corollary}
        Dado um grafo $G = (V, E)$ com $|V| = n$, um subgrafo \textbf{gerador} $F$ do grafo $G$ é uma \textbf{floresta perfeita} se
        \begin{enumerate}
            \item $F$ for uma floresta.
            \item todos os graus de todos os vértices de $F$ forem ímpares.
            \item cada uma das sub-árvores de $F$ é um subgrafo induzido de $G$.
        \end{enumerate}
    \end{corollary}

    \begin{definition}
        \begin{itemize}
            \item Um \textbf{emparelhamento} é um conjunto de arestas não-adjacentes.
            \item Um emparelhamento é dito \textbf{máximo} se possui o maior número possível de arestas.
            \item Um emparelhamento máximo é dito \textbf{perfeito} se suas arestas cobrem todos os vértices do grafo.
        \end{itemize}
        
    \end{definition}

\end{frame}


\begin{frame} % SLIDE 4
    \frametitle{}

    \begin{corollary}
        Dado um grafo $G = (V, E)$ com $|V| = n$, podemos afirmar que ele contém uma floresta perfeita se
        \begin{enumerate}
            \item $n$ for um número par.
            \item $G$ for conexo.
        \end{enumerate}
    \end{corollary}

\end{frame}

\section{Primeira prova}

\begin{frame} % SLIDE 3
    \frametitle{}

    \begin{lemma}
        Se $G$ for um grafo conexo com $n$ vértices, sendo $n$ par, e não for uma árvore, então $G$ possui uma árvore geradora com pelo menos dois vértices de grau par.
    \end{lemma}

    \begin{proof}[Prova]
        Seja $T$ uma árvore geradora de $G$.
        Suponha $T$ com todos os vértices de grau ímpar.
        Como $G$ não é uma árvore, existe ao menos uma aresta $e \in E(G \backslash T)$.
        Suponha que ela conecte os vértices $u$ e $v$.
        Considere a árvore $T^{*}$ formada pela adição de $e$ a $T$ e a remoção da aresta $(u, w)$, onde $w$ é o único vizinho de $u$ no caminho entre $u$ e $v$ em $T$. O grau de $v$ e $w$ agora é par, e a árvore geradora $T^{*}$ satisfaz o lema.
    \end{proof}

\end{frame}


\begin{frame} % SLIDE 3
    \frametitle{}

    \begin{theorem}
        Se $G$ é um grafo conexo com $n$ vértices e $n$ é um número par, então $G$ possui uma floresta perfeita.
    \end{theorem}


\end{frame}

\begin{frame}
    \frametitle{}

    \begin{proof}[Prova (por indução)]
        \begin{enumerate}
            \item Se $G$ for uma árvore com todos os vértices de grau ímpar, então $G$ é uma floresta perfeita.
            \item Se $G$ possuir ao menos um vértice $w$ de grau par, então há um número par de sub-árvores em $w$, sendo que ao menos uma delas contém um número par de vértices. Suponha que $G_{1}$ seja o subgrafo induzido de $G$ nesta sub-árvore e, portanto, possua ordem par. Então, $G_{2}$, o subgrafo induzido por $w$ e todas as suas demais sub-árvores, também possui ordem par. Como $G_{1}$ e $G_{2}$ são conexos, podemos dizer, por indução, que há uma floresta perfeita $F_{1}$ em $G_{1}$ e outra, $F_{2}$, em $G_{2}$. Por conseguinte, $F = F_{1} \cup F_{2}$ é uma floresta perfeita em $G$.
            % floresta + floresta = floresta
            \item Se $G$ não é uma árvore, então, pelo Lema, $G$ possui uma árvore geradora $T$ com pelo menos dois vértices de grau par. Por [2], sabemos que $T$ possui uma floresta perfeita e, por conseguinte, $G$ também a possui.
        \end{enumerate}
    \end{proof}

\end{frame}

\section{Segunda prova}


% \section{Referências bibliográficas}

% \begin{frame} % SLIDE 53
%     \frametitle{Referências bibliográficas}
%     \footnotesize{
%     \begin{thebibliography}{99} % Beamer does not support BibTeX so references must be inserted manually as below
%     % \bibitem[Smith, 2012]{p1} John Smith (2012)
%         \bibitem[Silberschatz, 2010]{p1}
%         Silberschatz, A., Korth, H. F., \& Sudarshan, S. (2010). Database system concepts (Vol. 4). New York: McGraw-Hill.

%         \bibitem[Rawat, 2017]{p2}
%         Rawat, U. (2017). Implementation of Locking in DBMS. Acessado a \date{25/11/2018} em https://www.geeksforgeeks.org/implementation-of-locking-in-dbms/.

%         \bibitem[Porfirio, 2013]{p3}
%         Porfirio, Alice \& Pellegrini, Alessandro \& Di Sanzo, Pierangelo \& Quaglia, Francesco. (2013). Transparent Support for Partial Rollback in Software Transactional Memories. 8097. 583-594. 10.1007/978-3-642-40047-6\_59. 

%         \bibitem[Poddar, 2013]{p4}
%         Poddar, Saumendra. (2003). SQL Server Transactions and Error Handling. Acessado a \date{25/11/2018} em https://www.codeproject.com/Articles/4451/SQL-Server-Transactions-and-Error-Handling.
%     \end{thebibliography}
%     }
% \end{frame}


% \begin{frame} % SLIDE 53
%     \frametitle{Referências bibliográficas}
%     \footnotesize{
%     \begin{thebibliography}{99} % Beamer does not support BibTeX so references must be inserted manually as below

%     \bibitem[Singhal, 2018]{p5}
%         Singhal, Akshay. (2018). Cascading Schedule | Cascading Rollback | Cascadeless Schedule. Acessado a \date{25/11/2018} em https://www.gatevidyalay.com/cascading-schedule-cascading-rollback-cascadeless-schedule/.

%         \bibitem[Pandey, 2018]{p6}
%         Pandey, Anand. (2018). Transactions and Concurrency Control. Acessado a \date{25/11/2018} em https://gradeup.co/transactions-and-concurrency-control-i-4c5d9b27-c5a7-11e5-bcc4-bc86a005f7ba.

% 	    \bibitem[Difference between, 2018]{p7}
%         Difference between. (2018). Difference between Deadlock and Starvation. Acessado a \date{25/11/2018} em http://www.differencebetween.info/difference-between-deadlock-and-starvation.

%     \end{thebibliography}
%     }
% \end{frame}

%------------------------------------------------

\end{document} 